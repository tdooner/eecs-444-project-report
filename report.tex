\documentclass{report}
\usepackage{geometry}
\usepackage{url}
\usepackage{cite}
\usepackage{graphicx}

\title{An Analysis of Security in Bitcoin, a Cryptographic Currency
Implementation}
\author{Ian Dimayuga, Tom Dooner, Will Wettersten \\
\url{{icd3,ted27,wbw20}@case.edu} \\ EECS444
\\ Case Western Reserve University}
\date{April 30, 2012}

\begin{document}
\maketitle
\section*{Cryptographic Currency}
% Tom
Cryptographic currency seeks to solve a timeless problem in economic systems:
how can value be transferred between entities in a way that is fair for both
entities? If the trade is disputed, what value remains with which party? And,
how can value be transferred through an intangible medium?

These are problems that current monetary systems cannot handle. Paper currency
is cumbersome and most-importantly cannot be electronically transferred. The
digital representation of money used by banks is actually just a clever system
of account withdrawals and credits in an agreed-upon order by banks.

\subsection*{Transferring Money, Today}
If an entity wished to transfer money to another entity today, they have a few
choices at their disposal. Some of these options include a check, a wire
transfer, the Automated Clearing House (ACH) system, or
PayPal\cite{wiki:wiretransfer}. All of these methods fail to employ mechanisms
to make them irreversible and impervious to unfair trades.

Wire Transfers rely on sophisticated networks of accounts at various banks, and
unless funds are quickly retrieved from the destination bank, the wire transfer
may be reversed by force of law. Wire transfer fraud is endemic to online
solicitations, and the FBI concludes that wire transfers and ACH transactions
cost businesses \$85 million USD per year\cite{wiretransferssuck}. 

Automated clearing houses are also go-to methods of stealing from unsuspecting
small businesses. The rate of ACH-based attacks on small businesses is growing,
along with the number of lawsuits against the banks that allow these attacks to
happen\cite{achsucks}. New laws place the loss in event of fraud squarely on the
bank. This is an unfortunate legal situation perpetuated by the lack of
technological solutions to the problem of fraud.

\subsection*{Enter Cryptographic Currency}
Cryptographic currency can solve the problem of fraud. If cryptographic currency
can ascribe value to a series of bits, then we can ensure that the appropriate
parties can be confident that their transactions were, in fact, initiated by the
appropriate entities with the right to the money.

However, this imposes new fascinating security problems. How can digital
currency prevent duplicate spending? What security must be taken to maintain the
private keys to your bank balance? Can someone design a system which is
impervious to interference from any single controlling entity?

But most importantly, what properties should digital cash fulfill? To answer
this question, we will take a brief detour into the works of David Chaum.

\section*{Properties of Cryptocurrency}
% Tom
In 1990, David Chaum, cryptologist and mathematician studying at UC Berkeley,
published techniques which allow for the minting of digital
currency.\cite{Chaum:Cash} Blind signature schemes, as they have come to be
called, allow the creation of digital currency which fulfills many desirable
properties of such a cash system. In 1991, Okamoto improved Chaum's algorithm
by identifying these properties of an ideal cryptographic currency and
implementing them:\cite{Okamoto:Cash}

\begin{enumerate}
\item \textbf{Independence}: The electronic cash must not depend on any physical condition.
\item \textbf{Security}: The electronic cash should prevent double-spending and illegal creation.
\item \textbf{Privacy}: Electronic cash transfers should not be traceable by a third party.
\item \textbf{Off-line Payment}: Electronic transfers should be able to occur off-line.
\item \textbf{Transferability}: The cash can be transferred to others.
\item \textbf{Dividability}: The cash can be divided into smaller pieces.
\end{enumerate}

In this paper, we will analyze a specific implementation of electronic cash:
Bitcoin. Bitcoin satisfies these properties, except for off-line payment. We
will specifically focus on Bitcoin's security model and how it has been
relatively error-free over the course of millions of dollars of transactions.

\section*{Introduction to Bitcoin}
\begin{figure}[h]
\begin{center}
\includegraphics[width=\textwidth]{images/bitcoin_v_time.pdf}
\caption{Bitcoin Exchange Rate Over Time}
\label{fig:bitcoinvalue}
\end{center}
\end{figure}
% Ian
In January 2009\cite{wiki:bitcoin}, Satoshi Nakamoto released his solution to the cryptographic currency problem. This currency, called Bitcoin, solves the problem of trusting centralized third parties such as banks or mints. Bitcoin is a peer-to-peer currency system, utilizing cryptographic proof and a network of nodes that must do computationally complex work in order to agree on the ``truth''\cite{Nakamoto:Bitcoin}.

The advantage of this system from a security standpoint is that attacks are computationally difficult. This difficulty arises from the fact that a network of honest nodes are continually working to generate computational proof of honesty, so a dishonest attacker would have to outpace their calculations. However, since Bitcoins are also mined using computing power, it is ideally more profitable to be an honest miner than a dishonest cheater.


% Include some graph of Bitcoin usage?

% From http://cryptome.org/0005/bitcoin-who.pdf
%   Talk about Kaminsky's analysis of Bitcoin's Security? (Maybe a subsection to
%    itself?)
%   Talk about the mysterious creator of Bitcoin? (And, how this is perhaps a
%    security benefit for the existence of the program)

\section*{Bitcoin Protocol Security}
% Ian
  \subsection*{Creating a Transaction}
    A Bitcoin is a chain of digitally-signed transactions. Each transaction documents a change in the coin's ownership, so its legitimacy can always be tracked back to its origin (the point when it was first mined). As indicated in Figure \ref{fig:coinchain}, a coin owner who wishes to transfer the coin combines a hash of the previous transaction with the payee's public key, and signs it with his private key. 

    \begin{figure}[h]
      \begin{center}
        \includegraphics{images/coinchain.png}
        \caption{Satoshi Nakamoto's Bitcoin. \cite{Nakamoto:Bitcoin}}
        \label{fig:coinchain}
      \end{center}
    \end{figure}
    
    This system allows for transactions to only be authorized by the owner of the private key associated with the public key at the end of the chain. This preserves ownership, but does not prevent double-spending on the part of the current owner. To achieve this, the network implements a distributed ``timestamp server'' whose goal is to perpetuate the one factual chronology of all transactions in the network. This creates a transaction which is easily verifiable using the public key from the previous transaction to prove the original ownership, and thus the right to pass on the coin. The node then broadcasts the transaction to the network.

  \subsection*{Proof-of-Work Chain}
    Each time a transaction is broadcast to the network, nodes must perform a computationally-difficult task to ``lock'' it into history. Nodes combine all current pending transactions into a ``block''. They combine this block with the hash of the previous block (to create the chain) and proceed to compute a nonce that, when appended to the end of the block, makes the block hash to a number smaller than a given minimum. This computation is exponential in complexity relative to the minimum specified, but can be verified simply by hashing the resulting block to compare it to the requirement.

    By doing this, nodes can agree on the block-chain history, preventing double-spending by preserving the transaction that has already occurred. To protect the blocks, nodes only accept the longest chain that currently exists in the network. This prevents fraudulent histories by making it computationally infeasible to create a longer chain as long as honest nodes hold the majority of the computing power.

    \begin{figure}[h]
      \begin{center}
        \includegraphics{images/blockchain.png}
        \caption{Proof-of-Work chain. Each `Tx' block is a pending transaction. \cite{Nakamoto:Bitcoin}}
        \label{fig:blockchain}
      \end{center}
    \end{figure}

\subsection*{Generation of Currency}
Bitcoin has a nontrivial security problem to solve in the creation of money: how
can currency be introduced fairly amongst all users in the network, such that no
user can unfairly generate currency. Since there is no central authority to seed
currency, there must be a secure distributed algorithm to introduce money into
the Bitcoin economy.

The brilliant method used by Bitcoin to introduce value into the economy
generates value while simultaneously incentivizing clients to participate in
computing the Proof-of-Work for the block chain. The cleverness is that when a
new block is discovered at the end of the Proof-of-Work chain, the next block is
initialized with a transaction granting its discoverer a number of Bitcoins.

Over time, the number of Bitcoins granted to the discoverer of the block chain
end diminishes until about 2040, when no future Bitcoins can be created. Figure
\ref{fig:totalcoins} shows the total number of Bitcoins that will be in
circulation at any point until about 2040, when the maximum of 21,000,000 coins
will be created.

\begin{figure}[h]
\begin{center}
\includegraphics[width=\textwidth]{images/quantity_v_time.pdf}
\caption{Total Number of Bitcoins In Circulation Over Time}
\label{fig:totalcoins}
\end{center}
\end{figure}

\section*{Bitcoin Wallet Security}
% Will
The access point to the Bitcoin network is the Bitcoin desktop client.  The client is a desktop
application that users can download to mine and trade Bitcoins.  The original client
was released by the creators of Bitcoin as an open source C++ project\cite{Andresen:source}, although
subsequent client programs have been released from various third parties.  

A Bitcoin user has two separate goals when protecting their wallet, namely protecting
against wallet loss and protecting against wallet theft.  The distinction between 
the two is that in the first, a user simply loses access to their wallet by deleting 
the wallet.dat file, losing their computer, or something equally ordinary.  The second 
goal, to protect against wallet theft, is much more complex an issue.

One simple approach to preventing theft that many Bitcoin users have taken is actually 
using a paper wallet to store their Bitcoin.  Several Bitcoin paper wallet utilities 
exist and are free to user on the internet.  To use them, a user typically downloads a 
key generator tool and generates a private key for any number of their Bitcoins.  The 
user then prints out the private key and makes sure that it is saved nowhere but 
physically on the piece of paper.  This way, a user can carry around any quantity of 
Bitcoins in their physical wallet.  When a user wants, they can import the Bitcoins 
either to their Bitcoin client or to a Bitcoin bank like Mt. Gox.

Another simple approach to preventing theft is to encrypt all or parts of the virtual 
wallet.  Many users encrypt their wallet.dat file, and the makers of Bitcoin are considering 
adding a utility in the next release of Bitcoin client that encrypts the private keys 
in the wallet.dat file.  Other users have their wallet.dat files in encrypted online 
repositories using services like Dropbox or Wuala in conjunction with an encryption 
program like 7-Zip or TrueCrypt.  Other users find even more success in encrypting the 
entire directory in which their Bitcoin client resides.

Still, however, the Bitcoin wallet represents a serious security problem for Bitcoin.  
By default, the wallet.dat file is not encrypted, so any hacker who can gain entrance 
into a victims computer can steal their Bitcoins simply by copying their wallet.dat file 
and then using then transferring the private keys to their account.  Although not intrinsic 
to Bitcoin theory, the wallet implementation is perhaps the foremost risk to Bitcoin.

\section*{Attacks Against Bitcoin}
% Will
As mentioned previously, Bitcoin is plagued by attacks to peripheral pieces of software 
that are not necessarily central to Bitcoin's concepts.  The virtual wallet is one example
of such a piece of software.  In a pattern that mimics the real world, criminals have
targeted virtual banks as well as virtual wallets.  Several such virtual banks exist on
the internet, many holding millions of dollars worth of Bitcoins.  Such large sums of 
money make virtual banks a popular target for attackers.

\subsection*{Bitcoinica Heist}
The most recent security blow dealt to the Bitcoin community occured in late March of
2012\cite{Goodwin:Bitcoinica}.  Eight users of the online Bitcoin trading market Bitcoinica were distrought one
Thursday morning when they found that large sums of Bitcoins were missing from their 
accounts.  As reports reached the ears of engineers at Bitcoinica, it became clear that
their had been an intrusion.  All in all, forty-six thousand bitcions were missing.  At
market, that amount would have fetched upwards of two-hundred and thrity thousand dollars.
Some digging revealed that the attackers had gained access to Bitcoinica's Linode servers.
Sure enough, on those servers were Bitcoin wallets.  Most of these wallets were unencrypted,
either out of ignorance, or so that the owners could easily automate trading.  To date there
are no leads on who stole the Bitcoins.

\subsection*{Allinvain Heist}
In a story of catastrophic personal loss for one Bitcoin user, virtual theives managed 
to get away with half a million dollars worth of Bitcoins from a single man in mid 2011\cite{Worstoll:Allinvain}.
At that point, Bitcoin was still in its infancy, making this one of the first big heists 
to rock the community.  Bitcoin user 'Allinvain' claims that twenty-five thousand Bitcoins,
at that time valued around five-hundred-million dollars, mysteriously dissappeared from 
his computer.  Investigation later revealed malware on his computer that had been recording
his keystrokes.  This could have provided the attacker remote access with which they could
have copied the bitcoins and then erased them from the hard drive.

One striking thing about this case is that there is no way of knowing who stole the Bitcoins, 
or worse, whether or not the 'victim' ever had them.  The theif could use them to buy 
something anywhere, even from the victim, without ever being caught.  This highlights one 
of the greatest strengths and weaknesses of Bitcoin; its anonymity;

\subsection*{Mt. Gox: The Ultimate Heist}
By far the largest heist in Bitcoin history is the Mt. Gox heist of June 2011\cite{Rashid:MtGox}.  Taken into
perspective the size of the currency and the impact of the heist, it is vertainly in the
running for most significant currency heist of all time.  The heist occured at the height
of the summer 2011 Bitcoin bubble, when the price of a Bitcoin had reached an astonishing
\$17.5.  Attackers breached the security of Mt. Gox, a prominant Bitcoin virtual bank,
and stole bitcoins from hundreds of the banks customers.  According to the bank, the attackers
made off with over 500,000 Bitcoins.  At market rate, they had stolen about nine million
dollars.

The only problem in this heist was a ripple effect caused by the attackers.  Being a young
currency, 500,000 Bitconis amounted to about 6\% of the total value of the market.
Unfortunately for the attackers, the affect of the heist and subsequent sell off was enough
to plunge the price of a Bitcoin to about a cent, making the total pot about \$5,000.  This
attack is interesting because of the enormous ripple it caused in the market.  In a show of
resiliance, the Bitcoin market recovered, with Bitcoins now trading around 7\$.

\section*{Conclusion}
Bitcoin is a fascinating suite of protocols, expertly designed to be
impregnable. The design of the generation of currency allows for an equitable
split of currency amongst users despite the lack of a central authority. The
use of a proof-of-work chain to cement transactions into an agreeable history is
a beautifully novel method to create a distributed consensus.

Over time Bitcoin has proven its resiliency on a protocol level. However, the
implementations of Bitcoin exchanges are extremely vulnerable to garden-variety
remote attacks. As Bitcoin exchanges learn to cope with these threats, Bitcoin's
protocol security will shine through to provide a solid distributed transaction
system for users throughout the world.

\bibliography{security}{}
\bibliographystyle{plain}
\end{document}
